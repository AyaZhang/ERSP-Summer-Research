\documentclass[dvips,12pt]{article}

% Any percent sign marks a comment to the end of the line

% Every latex document starts with a documentclass declaration like this
% The option dvips allows for graphics, 12pt is the font size, and article
%   is the style

\usepackage[pdftex]{graphicx}
\usepackage{url}

% These are additional packages for "pdflatex", graphics, and to include
% hyperlinks inside a document.

\setlength{\oddsidemargin}{0.25in}
\setlength{\textwidth}{6.5in}
\setlength{\topmargin}{0in}
\setlength{\textheight}{8.5in}

% These force using more of the margins that is the default style

\begin{document}

% Everything after this becomes content
% Replace the text between curly brackets with your own

\title{\textbf{Aesthetic/Utility Classification and Prediction Algorithm for Amazon Products}}
\author{Linda Wogulis, Yijun Zhang}
\date{\today}

% You can leave out "date" and it will be added automatically for today
% You can change the "\today" date to any text you like


\maketitle

% This command causes the title to be created in the document

\section{Background}

% An article style is separated into sections and subsections with 
%   markup such as this.  Use \section*{Principles} for unnumbered sections.


\section{Project Description}



\section{Challenges}

At this stage, there are four expected challenges that we identify:

\paragraph{Manually labeling data}
We will be using Amazon product data provided by Prof McAuley. The Amazon product dataset contains product reviews and metadata from Amazon, including 143.7 million reviews spanning May 1996 - July 2014.//CITE.//NOT SURE. Since we are interested in discerning the difference between a product being sold for aesthetic or utility, we need to manually label (need to confirm: label what). Only one or two categories will be chosen for us to train and test our model. Still, this can be a very exhausting task. 

\paragraph{Defining the problem in the right way}
Whether a product is sold for aesthetic or utility is more of a subjective question. Therefore, we need to define aesthetic and utility to classify products. In addition, we also need to determine a proper representation of the products, which can be binary, or on a scale of one to ten, where 10 means the product is sold purely for utility.

\paragraph{Scaling models to large datasets}
In order to develop a scalable model, we may need to learn techniques to handle large datasets.

\paragraph{Statistical significance}
To determine whether our finding has statistical significance, we need to learn to perform statistical hypothesis testing on our model.

\section{Tools/Knowledge to learn}

\paragraph{Regression} 
Regression is an approach to analyze a training set of data to develop a hypothetical relationship between features of products and parameters to predict real-valued outputs, ie. people's tendency of buying the product for aesthetic or utility. Specifically, we are going to look at linear regression with multiple variables, a model which best describes our problem.
	
\paragraph{Classification}
classification or regression?
	
\paragraph{Bag-of-words}
Bag-of-words is a model representation in processing natural language by treating text as merely a collection of words, neglecting grammar and punctuations. By looking at customer reviews as a bag of words, we seek to identify keywords that are indicative of product appearance or utility based on frequency, and use them as features for regression and classification.
	
\paragraph{Sentiment analysis}
Sentiment analysis aims to identify subjective opinions based on text. We want to learn about sentiment analysis to associate keywords with product properties of aesthetic or utility.

\section{Timeline}
The project period is 8 weeks. Lists of tasks per stage:

\begin{itemize}
	
	\item Week 1 \& 2: Preparation
	\begin{itemize}
		\item Learn about basic models for natural language
		\item Learn about regression using text			
		\item Reproduce standard techniques to model data
	\end{itemize}
	
	\item Week 3 \& 4: Data processing and labeling
	\begin{itemize}
		\item Develop proper definitions of status/utility for our data
		\item Manually label the data using the definitions above
	\end{itemize}
	
	\item Week 5 \& 6: Modeling
	\begin{itemize}
		\item Run experiments to try out different models and features
		\item Identify those that are effective at predicting the labeled data
	\end{itemize}
	
	\item Week 7 \& 8: Analysis
	\begin{itemize}
		\item High level analysis of our findings
	\end{itemize}
\end{itemize}




\begin{thebibliography}{99}

\bibitem{gonzalez2012} Jonay I. Gonz\'{a}lez Hern\'{a}ndez, 
Pilar Ruiz-Lapuente,	
Hugo M. Tabernero,	
David Montes,	
Ramon Canal,	
Javier M\'{e}ndez	
and Luigi R. Bedin,
{No surviving evolved companions of the progenitor of SN1006},
Nature, {\bf 489}, 533-536 (2012).

\end{thebibliography}


\end{document}
